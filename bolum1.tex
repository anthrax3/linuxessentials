\chapter{Bilgisayarlar, Yazılım ve İşletim Sistemleri}
\label{chap:bolum1}
\paragraph{Amaçlar}
\begin{itemize}
 \item Temel bilgisayar donanım bilgisini edinmek
 \item Farklı işletim sistemlerinin farkında olmak ve bu sistemlerin farklı yönleri ile benzer yönlerini tayin etmek
 \end{itemize}
 
\paragraph{Önceden Bilinmesi Gerekenler}
\begin{itemize}
 \item Temel bilgisayar bilgisi işe yarar olacaktır.
 \end{itemize}

\begin{section}{Bilgisayar da neyin nesi?}
Bilgisayarın ne olduğunun ayrıntılarına girmeden önce, bilgisayar camiasının dikkate değer kişilerinden birkaç alıntı ile işe başlayalım.
\begin{quote}{Esasında Birleşik Devletler'de, gizlenmiş araştırma laboratuvarlarında yarım düzine kadar büyük bilgisayarlardan olsaydı; bu, ülke olarak ihtiyaç duyduğumuz şeylerin çaresine bakabilirdi. Howard H. Aiken, 1952}
\end{quote}
Howard Aiken bilgisayar alanında bir öncüydü ve IBM' nin ilk bilgisayarı "Harvard Mark I" in tasarımcısıydı. Modern anlayışa göre inşaa edilmiş ilk bilgisayarlar İkinci Dünya Savaşında şifrelenmiş mesajları deşifre etmek için ya da zor hesaplamaları yapmak için yapılmıştı; büyük, karmaşık ve hataya yatkınlardı. Bugünlerde bilgisayarlarda bulunan transistör ya da entegre devreler henüz icat edilmemişti. Bu zamanlarda aydınlığa kavuşan şey, savaşın hemen sonrasında ortaya çıkan birtakım temel varsayımlarla oluşturulan "bilgisayar" olarak kabul edilebilecek bir cihazın varlığıydı.
\begin{itemize}
 \item Bilgisayar veriyi 'otomatik olarak' çalıştırılan komutların sırasına göre işler,
 \item Programlar şartlı çalışmalara ve döngülere izin vermelidir,
 \item Bilgisayarın çalıştırdığı programı değiştirmek veya yerine başka bir şey yerleştirmek mümkün olmalıdır
 \end{itemize}

Örnek olarak, çoğu teknolojik cihaz - televizyon setlerinden dijital kameralara çamaşır makinesine ya da arabalara kadar - bugünlerde neredeyse küçük bir bilgisayar sayılabilecek programlanmış kontrol birimleri içerir. Böyle olmasına rağmen bu cihazları "bilgisayar" olarak görmeyiz, çünkü bu cihazlar sadece düzenlenmiş ve değiştirilemez programları çalıştırırlar.  Buna karşın bir hesap makinesi "verileri işlemek" için kullanılabilir ama - eğer "programlanabilir bir hesap makinesi" kadar pahalı değilse - bu otomatik olarak olmaz; bir insan tuşlara basmalıdır.

1950'nin başlarında, bilgisayarlar insanların ancak araştırma kurumlarında görmeyi bekleyeceği – Aiken'in öngördüğü gibi- yüksek derecede özelleştirilmiş cihazlardı. Zamanın bilim-kurgu filmleri gizemli dönen çarklarla dolu dolapların olduğu koridorları gösteriyordu. Tam olarak 70 yıl bile olmadan bu görüntü önemli bir ölçüde değişti. 
\footnote{Bu metinde belirtilen alıntının aslında Thomas J. Watson, IBM'in CEO'su, 1943'te “Dünya pazarında sadece beş bilgisayar için yer vardır.” sözleriyle birlikte söylendiği tasvir edilir. Ne yazık ki bu hiçbir zaman doğrulanamamıştır. Ve eğer 1943'te gerçekten bunu söylediyse, bu en azından 10 yıl için geçerli olurdu.}
\begin{quote}
{Kimsenin evinde bir bilgisayara sahip olmasına gerek yoktur. Ken Olsen, 1977}
\end{quote}

Ken Olsen bir bilgisayar üreticisi olan ve 1970' lerdeki “küçük” kavramının “havalandırmalı bir makine odasına ve güç ünitesine ihtiyaç duymayan ve bir milyon dolardan daha ucuza mal olan” anlamına geldiği zamanlarda “küçük” bilgisayarların geliştirilmesinde öncü bir firma olan  DEC (Digital Equipment Corporation) \footnote{DEC 1998 yılında Compaq tarafından alındı, Compaq ise 2002' de Hewlett-Packard tarafından alındı.} firmasının yöneticisi idi. Donanım teknolojisinin gelişmesiyle 1970' lerin sonuna doğru “küçük” kavramı “iki insan tarafından taşınabilir” anlamına gelmeye başladı.

DEC Linux camiası için önemlidir çünkü Unix - kendi oluşumundan 20 yıl sonra Linus Torvald' a Linux' u başlatmak için ilham veren sistem - ilk olarak DEC PDP-8 ve PDP-11 bilgisayarları üzerinde geliştirilmiştir.

Ayrıca 1970'ler ilk “ev bilgisayarları” nın varlığını gördü. Bu günümüzün kişisel bilgisayarlarıyla karşılaştırılamaz çünkü insanlar evde kendileri bilgisayarlarını lehimlemek durumundaydı (ki bu zamanlarda fiziksel olarak imkansız bir durum olurdu) ve bu bilgisayarlar çok nadir olarak kullanılabilir bir klavye ve iyi bir ekran ile geliyordu. Bu bilgisayarlar genellikle bir tamircinin eski kullandığı malzemelerden yapılıyordu, bir elektrikli tren setinden geriye kalanlar gibi, çünkü gerçekte çok kullanışlı değillerdi. Buna rağmen, bizim önceden yaptığımız tanımımıza göre “bilgisayar” olarak adlandırılıyorlardı çünkü özgürce programlanabiliyorlardı, bu zahmetli bir şekilde her şeyi tuşlardan girmek ya da (eğer o kadar şanslıysanız) bir ses kaset teypinden yüklemek anlamına gelse bile. Yine de bu bilgisayarlar tümüyle ciddiye alınmıyorlardı ve Ken Olsen'in cümlesi sık sık yanlış yorumlandı. O hiçbir şekilde küçük bilgisayarlara karşı değildi (hatta işi bu bilgisayarları satmaktı). Onun anlamadığı şey bütün ev işlerinin (ısıtma, aydınlatma, eğlence ve bunun gibi şeyler) bir bilgisayar tarafından kontrol edilmesi fikriydi – bu fikir o zamanlar sadece teoride vardı ama günümüzde oldukça uygulanabilir bir durumda.

1970' lerin sonunda ve 1980' lerde “ev bilgisayarları” parçalar topluluğundan kullanmaya hazır  cihazlara (“Apple II” ya da“Commodore 64” gibi isimler hala bu işin içinde olan eski üyelerimize tanıdık gelebilir) dönüştü ve ofislerde de bu cihazlardan bulunmaya başladı. IBM tarafından ilk kişisel bilgisayar 1981' de duyruldu ve Apple ilk “Macintosh” u 1984' te piyasaya sundu. Bunların dışındakiler ,söylenildiği gibi, tarihte kaldı ama bilgisayar dünyasının sadece kişisel bilgisayarlardan ya da Mac' lerden oluşmadığı unutulmamalıdır.  Devasa, odaları dolduran eski bilgisayarlar hala bulunmaktadır ama bu durum gitgide azalmakta ve büyük gruplarda günümüz bilgisayarlarından oluşmaktadır. Bununla birlikte temel ilke Howard Aiken' in zamanında beri değişmedi: Bilgisayarlar hala veriyi koşul ve döngüler içerebilen değişebilir programlara göre otomatik olarak işleyen cihazlardır. Ve işler bu şekilde yürümeye devam edecek gibi görünüyor.
\paragraph{Alıştırmalar}{
\begin{itemize}
 \item İlk kullandığınız bilgisayar neydi? Ne çeşit işlemci içeriyordu, ne kadar belleği vardı ve sabit diski ne kadar büyüktü (eğer bir sabit diski bulunuyorduysa - eğer sabit diski bulunmuyorduysa veri nasıl kalıcı olarak depolanıyordu)?
\end{itemize}}
\end{section}

\begin{section}{Bir Bilgisayarın Parçaları}

Hadi bir bilgisayarın (ya da daha açık olmak gerekirse IBM-uyumlu bir kişisel bilgisisayarın) içine bakma fırsatını yakalayalım ve burada bulmamızın mümkün olduğu parçaları inceleyelim:
\paragraph{İşlemci}{İşlemci (“CPU” Türkçe anlamıyla “Merkezi İşlem Ünitesi”) bilgisayarın çekirdeğidir: Burası bilgisayarı bilgisayar yapan, program kontrolündeki verilerin işlendiği yerdir. Bugünün işlemcileri genelde birkaç "çekirdek" içerir, bunun anlamı işlemcinin ana parçaları birden fazladır ve bağımsız olarak (bağımsız işlem yapabilme bilgisayarın işlem hızını ve buna bağlı olarak performansını arttırır) işlem yapabilirler.  Genelde hızlı bilgisayarlar birden fazla işlemci içerirler. Bilgisayarlar normalde Intel ya da AMD işlemcilerine (detaylarda farklılık gösterebilir ama aynı programları çalıştırabilirler) sahiptir. Tabletler ve akıllı telefonlar genel olarak ARM işlemcilerini kullanırlar. Bunlar diğerleri kadar güçlü değildirler ama enerjiyi daha az tüketirler. Intel ve AMD işlemcileri direkt olarak ARM için hazırlanmış programları çalıştıramazlar, aynı durum ARM işlemcileri içinde geçerlidir.}
\paragraph{Bellek}{Bir bilgisayarın çalışma belleğine “RAM” (ya da random-access memory/rastgele erişimli bellek, buradaki rastgele bir gelişigüzelliği değil isteğe bağlı erişimi ifade eder) denir. Bu bellekte sadece işlenen veriyi değil aynı zamanda çalıştırılan programın kodunu tutar.}

Bu fikir Howard Aiken' in döneminden bilgisayarın öncülerinden olan John von Neumann' a aittir. Kod ve veri arasında hiç farklılığın olmadığını öngörür – bu programların bizim adresleri ya da yemek tariflerini değiştirdiğimiz gibi kodu değiştirebileceği anlamına gelir. (Eski günlerde, programlar kabloların yeri değiştirilerek ya da delikli kartlar oluşturularak yazılırdı ve bu programlar değiştirilemezdi.)

Bugünün bilgisayarları normalde 1 gibibyte bellek belki de daha fazlasını kullanıyor. 1 gibibyte 2**30'a eşdeğerdir, bu 1,073,741,824 
\footnote{İnsanlar genelde bunu “gigabyte” olarak söyler ama gigabyte normalden yüzde 7 daha azdır.}
byte eder ve gerçekten çok büyük bir sayıdır. Karşılaştırma olarak düşünürsek: Harry Potter ve Ölüm Yadigarları kitabı her sayfa karakter, boşluk ve noktalama işaretlerinden oluşan 1700 karakterden fazla olmak üzere yaklaşık olarak 600 sayfa içerir. Bu muhtemel olarak bir milyon karakter demektir. Bununla birlikte bir gibibyte 1,000 tane Harry Potter kitabına tekabül eder ve eğer sadece genç bir büyücünün kahramanlıkları ile ilgili değilseniz 1,000 kitap inanılmaz bir kütüphane oluşturur.
\paragraph{Ekran kartı}{Çok uzak olmayan bir geçmişte insanlar bilgisayarlarda bir çıktıyı oluşturmak için elektrikli bir daktiloyu kullanabiliyorsa mutlulardı. Eski ev bilgisayarları televizyon setlerine bağlılardı ve genelde berbat denilebilecek resimler oluştururlardı. Diğer elde ise bugün en basit “akıllı telefonlar” bile oldukça etkileyici grafikler sunuyorlar ve şu anda kullanılan kişisel bilgisayarlardaki ekran kartları 1990' larda 
\footnote{Bu arada bütün teşekkürler harika bilgisayar oyunlarının tükenmeyen popülerliğine gitsin. Kim bilgisayar oyunlarının işe yaramaz olduğunu düşünüyorsa bir dakikasını bunun üstünde düşünerek geçirmelidir.}
pahalı bir spor arabaya ve küçük bir ev değerinde bir maliyete sahip olurdu. Bugünün sloganı “3D hızlandırma”, ki bu durumda ekran gerçekte 3D olarak çalışmaz (ki bu bile gitgitde moda olmaya başladı) bilgisayarın içinde işlenen grafikler sadece sağ, sol, yukarı ve aşağıyı içermez – bilgisayar ekranında görülebilen yönler- aynı zamanda ön ve geri kısımları içerir. Görüntü gerçeklik oyunları için bir canavarın bir duvarın önünden mi arkasından mı çıkacağı çok önemlidir bununla birlikte görünür olsun ya da olmasın modern ekran kartlarının amacı bilgisayar işlemcisini diğer işler için serbest bırakmaktır. Güncel ekran kartları kendi işlemcilerine sahiptir, bunlar bilgisayarın kendi işlemcilerinden daha hızlı çalışırlar ama onlar kadar genel olarak kullanışlı değildirler.}

Çoğu bilgisayar ayrı bir ekran kartı içermez çünkü bu bilgisayarların grafik donanımları işlemcinin bir parçasıdır. Bu bilgisayarı daha küçük, ucuz, sessiz yapar ve bunlar enerjiyi daha iyi kullanır ama grafik performansı çok iyi değildir. Eğer en yeni oyunları oynama bağımlısı değilseniz bu sizin için gerçek bir sorun teşkil etmez.
\paragraph{Anakart}{Anakart genel olarak dikdörtgen biçimindedir, bilgisayarın işlemcisinin, belleğinin ve grafik kartının takılı olduğu ve örneğin sabit disklerin, yazıcıların, bir klavyenin ve farenin ya da ağ kablolarının ve elektronik olarak gerekli olan her şeyin kontrolcüsünün bulunduğu bir levhadır. Bilgisayarlarda bulunan anakartlar birçok çeşitli boy ve renklerde 
\footnote{Gerçekten! Ama yinede kimse bilgisayarının anakartını rengine göre seçmemelidir.}
olabilir, örneğin oturma odasında bulunan bir video kaydedici olarak kullanılan küçük ve sessiz bir bilgisayarın anakartı ile çok fazla RAM'e ihtiyaç duyan ve birden fazla işlemcisi bulunan büyük sunucuların anakartları farklı boyut ve renklerde olabilir.}
\paragraph{Güç Kaynağı}{Bir bilgisayar çalışmak için elektriğe ihtiyaç duyar, ne kadar elektriğe ihtiyaç duyduğu sahip olduğu bileşenlere bağlıdır. Güç kaynağı 240 V AC kaynağını bilgisayarın ihtiyaç duyduğu daha düşük değerlerdeki DC voltajına indirmek için kullanılır. Güç kaynağı bütün parçalar için yeterli elektriği üretecek şekilde seçilmelidir. Güç kaynağının bilgisayara verdiği elektriğin çoğu er ya da geç ısıya dönüşecektir bu yüzden bilgisayarlarda soğutma çok önemlidir. Basit tasarımlarda, genelde bir ya da iki fan pahalı elektronik parçalara hava üfler ya da sıcak havayı dışarı atar. Uygun bir tasarımla bilgisayarları fana ihtiyaç duymayacak şekilde yapmak mümkündür bu şekilde bilgisayarlar çok sesssiz çalışırlar ama bu tür bilgisayarlar hem oldukça pahalı hem de çok hızlı değillerdir (İşlemci ve ekran kartların hızlı olması genelde sıcak olması anlamına gelir).}
\paragraph{Sabit Diskler}{Bilgisayarın belleği anlık çalışan süreçlerin (belgeler, dökümanlar, web sayfaları, geliştirilen programlar, müzik ve videolar,... - ve tabiki verinin üzerinde çalışan programlar) verilerinin depolandığı bir yerken kullanılmayan veriler sabit diskte depolanır. Bunun ana sebebi sabit disklerin genelde bilgisayarların belleklerinden çok daha fazla veri depolayabilmesidir – günümüzde sabit diskler terabytelar cinsinden ölçülmektedir.}

Boyuttaki bu genişleme hızda yavaşlama ile doğru orantılı ilerler. Bellek erişim zamanı nanosaniyeler ile ölçülürken sabit disklerde bu durum milisaniyelerle ölçülür. Bu, bir metre ile 1000 kilometre arasındaki farka eşdeğerdir.

Geleneksel olarak, sabit diskler manyetik madde içeren dönen plakalar içerir. Okuma / yazma kafası bu maddeyi değişik yerlerden manyetize edebilir ve depolanmış verileri okuyabilir. Bu plakalar dakikada 4,500 ile 15,000 defa döner ve okuma/yazma kafası ile plaka arası bir dakikadır (en fazla3 nanosaniye). Bu sabit disklerin çok hassas olduğu anlamına gelir, çünkü eğer okuma/yazma kafası disk hala çalışırken plaka ile iletişime geçerse kesintiye uğramış kafa kırılır ve disk bozulur.

Taşınabilir bilgisayarlar için olan son moda sabit diskler, bilgisayarın düştüğünü anlayabilecek ve o anda sabit diskte oluşabilecek zararı önlemek için bilgisayarı kapatacak hızlandırılmış sensörlere sahiptirler.

Son moda SSD'ler ya da katı-durum diskleri, manyetik plakalar kullanmak yerine depolamak için “flash belleği” kullanır – bu içindekileri elektrik olmadan tutan bir bellek çeşididir. Katı-durum diskleri manyetik sabit disklerden daha hızlıdır fakat görece olarak daha pahalıdırlar. Bununla birlikte hareket eden parçaları yoktur, düşmeye dayanıklıdırlar, sabit disklere göre enerjiyi tasarruflu kullanırlar. Bu özellikleri taşınabilir bilgisayarlar için çok iyidir.

Flash bellekler belirli sayıda yazma işlemine tabii tutulabilir. Ölçümler bunun pratikte bir sorun teşkil etmediğini göstermiştir.

Bir sabit diski (manyetik ya da SSD) bilgisayara bağlamak için çeşitli yöntemler vardır. Şu anda en kullanılanı “serial ATA” (SATA) olarak adlandırılır, ayrıca “IDE” olarak da bilinir. Ayrıca sunucular SCSI ya da SAS disklerini kullanırlar. Harici diskler için, USB ya da eSATA (SATA' nın sağlam bağlantısı noktası olanı) kullanılabilir.

Sözü açılmışken: Gigabytelar ile gibibytelar (ya da terabytelar ile tebibytelar arasında) arasındaki fark en çok sabit disklerde fark edilir. Örneğin 100 GB disk alırsınız ama bilgisayarınıza bağladığınızda bir bakarsınız 93 GB görünüyor?! Bununla birlikte diskiniz arızalı değildir (çok şanlısınız) disk sürücüsünü üreten firma gigabyteları kullanırken bilgisayarınız muhtemel olarak alanları gibibyte cinsinden hesaplıyordur.
\paragraph{Optik Sürücüler}{Sabit disklerin dışında bilgisayarlar genelde okuma ve genelde yazma işlemlerini yapabilen optik sürücüleri içerirler. (CD-ROM, DVD-ROM ve Blu-ray diskler) Mobil cihazlar bazen optik sürücüler için yeterli fiziksel hacme sahip olmazlar ama bu optik sürücüleri desteklemedikleri anlamına gelmez. Optik medyalar – isim verilere erişmek için kullanılan lazerden gelir- genellikle yazılım ve ürünlerin (müzik ya da filmlerin) dağıtımında kullanılırlar. Bu medyaların dağıtımı için internet etkin olarak kullanılmaya başlandığı için optik medyaların önemi azalmaktadır.}
Eski zamanlarda optik medyalar yedekleme için kullanılıyordu ama bugünlerde bu çok anlamsızdır çünkü CD-ROM ortalama olarak 700 MB lık ve DVD-ROM ortalama olarak 9 GiB veri tutabilirler. Bu sebeple 1 TBlık bir veri için tam bir yedekleme 1000 CD ya da 100 DVD gerektirir. (Blu-ray diskler 25 GB a kadar veri tutabilir fakat blu-ray diskleri okuyan okuyucu mekanizmalar oldukça pahalıdır.)
\paragraph{Ekran}{Eski filmlerde hala yeşil renkli bilgisayar ekranlarını görebilirsiniz. Gerçekte yeşil ekranların hepsi ortadan kaybolmuş durumdadır, renkler oldukça gözde bir durumdalar ve yeni ekranlar eskiden sahip olduğumuz CRT ler (tüplü monitörler) gibi devasa değiller. Sıvı kristal (LCD) teknolojisine dayanan ince ve şık monitörlere sahibiz artık. LCD' ler kendilerini masada daha az yer kaplamanın avantajıyla sınırlamıyorlar, aynı şekilde ekran ışıkları titreşmiyor ve kullanıcıları muhtemel zararlı radyasyonlara maruz bırakmıyor. Buna her yönden kazanma durumu da diyebiliriz.  Değişik açılardan bakıldığında renk değişikliklerinin olması ve ucuz cihazlarda kötü ışık durumları birkaç dezavantajı bulunur.}

Tüplü monitörlerde kullanılmayan bir resmin uzun bir süre ekranda kalmamasına dikkat edilirdi çünkü resim ekrana yazılabilir ve kalıcı bulanık bir fon olarak ekranda kalabilirdi. Ekran koruyucular belirli bir süre ekran kullanılmadığında şirin bir canlandırma ekrana gelir ve bu “yazılma” sorununu engellerdi. (klasik olan bir akvaryum balığıydı). LCD' ler artık bu sorunu yaşamamaktadır ama ekran koruyucular hala dekoratif olarak bulunmaktadır.

LCDler akıllı telefon boyutundan duvar boyutuna kadar bütün boyutlarda bulunabilir, en önemli özelliği çözünürlükleridir, bilgisayarlar genelde 1366 x 768 (yatay x dikey) ile 1920 x 1080 arasında bir çözünürlük sunarlar. (Daha düşük ve daha yüksek çözünürlükler mümkündür ama ekonomik ya da görsel olarak gerekli değillerdir.) Bilgisayarların büyük bir kısmı genişletilmiş çalışma ortamlarında birden fazla ekranı desteklerler.

Ayrıca bugün genel olarak yüksek çözünürlüklü televizyona karşılık gelen 16:9 ölçüsü bulunur – bilgisayarların büyük bir kısmı televizyon izlemek için kullanılmadığı için aslında bu saçma bir gelişmedir. Daha uzun ama dar ekranlar (4:3 gibi formatlar) sık kullanılan programlar için daha uygun olurlar.
\paragraph{Diğer Bileşenler}{Bilgisayara bizim anlattığımız bileşenlerden daha fazlasını bağlamanız mümkündür. Yazıcılar, tarayıcılar, kameralar, televizyon alıcıları, modemler, robotik kollar, komşularınızı rahatsız edecek küçük füze atıcılar ve bunun gibi şeyler. Bu listenin gerçekten bir sonu yok ve bütün değişik cihazları burada anlatamayız. Yine de bu birkaç gözlem yapamayacağımız anlamına gelmez.}
\begin{itemize}
 \item Bir tane güzel yeniliklerden biri, örnek olarak, bağlantıların basitleştirilmesidir. Neredeyse bütün değişik sınıftaki cihazlar kendi arayüzlerini kullanıyorken (yazıcılar için paralel arayüzler, modemler için seri arayüzler, klavye ve fareler için PS/2 arayüzü, tarayıcılar için SCSI, …) bugünlerde çoğu cihaz USB' yi (universal serial bus) kullanıyor. USB kısmen güvenilir ve kabul edilebilir bir hıza sahip olmakla birlikte bilgisayar çalışırken tak-kullan özelliğini destekler.
 \item Bir diğer yenilik bileşenlerin kendi içlerinde daha akıllı olması ile ilgili. Önceden pahalı yazıcılar bile elektrikli daktiloların kabul edilebilir derecede IQ derecesine sahip olan aptal cihazlardı ve programcıların istenen doğru çıktıyı alabilmek için yazıcıya tam olarak doğru kodu oldukça dikkatli bir şekilde göndermesi gerekiyordu. Bugün yazıcılar (en azından iyi yazıcılar) programcılar için daha az sorun yaratacak şekilde kendi destekledikleri özgün programlama dillerine sahipler ve kendi içlerinde bir bilgisayar sayılabilirler. Bu durum çoğu diğer bileşen için aynıdır.
\end{itemize}
\paragraph{Alıştırmalar}{
\begin{itemize}
 \item Bilgisayarınızın içini açın (tercihen bir öğretmen ya da bir ebeveyn gözetiminde ve önce elektrik bağlantılarını kapatmayı unutmayın!) ve işlemci, bellek, anakart, ekran kartı, güç kaynağı ve sabit disk gibi önemli parçaları bulmaya çalışın. Bilgisayarınızın hangi parçalarından burada bahsedilmedi?
\end{itemize}}
\end{section}

\begin{section}{Yazılım}

Bilgisayarın donanımı, içerdiği teknik parçalar, önemli olduğu gibi yazılımı da
\footnote{Donanımın tanımı: “Bilgisayarın vurulabilen parçalarına denir.” (Jeff Pesis)}, çalıştırılabilir programlar, oldukça önemlidir. Bu yaklaşık olarak üç kategoriye bölünebilir.

Firmware bilgisayarın anakartında depolanır ve eğer uygun değilse değiştirilip yerine başka bir firmware kullanılabilir. Bu bilgisayarı açtıktan sonra bilgisayarı tanımlı olan bir duruma getirmek için kullanılır. Genelde saati ayarlamaya yarayan ya da anakarttaki bazı özellikleri açıp kapatmaya yarayan bir kurulum modunu uyandırmak için bir yol bulunur.

Kişisel bilgisayarlarda bu yazılım BIOS (Temel Girdi/Çıktı Sistemi) olarak adlandırılır, daha yeni sistemlerde bunun adı EFI olarak geçer.

Bazı anakartlar normal Linux' dan daha hızlı açılan küçük bir Linux sistemi yüklü olarak gelirler, bu sadece Windows' u açmaya gerek kalmadan internette gezinme ya da DVD izleme gibi sınırlı işlevler görür.

İşletim sistemleri bilgisayarı kullanılabilir bir cihaz yapar: İşletim sistemi bellek, sabit disk, ayrı çalışan programların işlemcide kullanabilecekleri zaman ve diğer bileşenlere erişim gibi bilgisayar kaynaklarının kullanımının nasıl olacağını yönetir. Programların başlamasına ve durmasına izin verir ve aynı bilgisayarda olan farklı kullanıcılar arasında bir ayırıcı görevi görür. Bunların dışında giriş seviyesinde bilgisayarın yerel bir ağa ya da internete katılmasını sağlar. İşletim sistemi görsel bir arayüz sunar ve bu kullanıcıların bilgisayarın nasıl çalıştığı ile ilgili bir fikir verir.

Yeni bir bilgisayar aldığınızda bu genellikle önceden kurulmuş bir işletim sistemi ile gelir: Kişisel bilgisayarlar Microsoft Windows, Mac bilgisayarlar OS X, akıllı telefonlar genelde Android (bir Linux türevi) kullanır. Bununla birlikte işletim sistemi donanıma firmware gibi bağlı değildir ve herhangi birisi bir yenisiyle değiştirilebilir. Örneğin çoğu kişisel bilgisayarlara ve Mac bilgisayarlara Linux kurabilirsiniz.

Ya da Linux'u var olan bir işletim sisteminin yanına ek olarak da kurabilirsiniz, bu bir sorun yaratmaz.

Kullanıcı-seviye programlar kullanışlı bir şeyler yapmanızı sağlarlar. Bu belge yazmak, resim çizmek ya da değiştirmek, müzik bestelemek, oyun oynamak, internette gezinmek ya da yeni bir yazılım geliştirmek olabilir. Bu programlar uygulamalar olarak adlandırılırlar. Bunlara ek olarak genelde işletim sisteminin size sağladığı bazı araçlar bulunur, bunlar bilgisayar ayarlarında değişiklik yapmanıza izin verirler. Sunucular genelde diğer bilgisayarlara web, posta ya da veritabanı gibi bazı hizmetler sağlarlar.
\end{section}


\begin{section}{En Önemli İşletim Sistemleri}
\begin{subsection}{Windows ve OS X}

İşletim sistemi denildiğinde çoğu insanın aklına Microsoft Windows geliyor otomatik olarak. Bu günümüzde bilgisayarlarının çoğunun Windows yüklü olarak satılmasından kaynaklanıyor. Bu kendi başına kötü bir şey sayılmaz, çünkü bilgisayar sahipleri ilk sistem yükleme işlemi için uğraşmak zorunda kalmazlar. Ama olaya başka bir açıdan bakarsak bu Linux gibi alternatif sistemlerin tanınmasında bir sorun yaratıyor.

Aslında bilgisayarı Windows yüklü olarak almak o kadarda zor değildir çünkü bilgisayarınızı Linux ile kullanmak isteyebilirsiniz ama bu durumda sıfırdan bir sistem inşa etmek durumunda kalırsınız. Teoride bakıldığında kullanılmamış bir Windows için bilgisayar üreticisinden paranızı geri alabilirsiniz ama hepimiz biliyoruz ki bugüne kadar parasını geri almayı başarabilen çok fazla kişi olmadı.

Günümüzde kullanılan Windows 1990'lı yılların standartlarına göre oldukça iyi bir işletim sistemi olan “Windows NT” nin  varisidir. (Windows 95 gibi önceki sürümler mevcut bulunan Microsoft işletim sistemi MS-DOS' un grafiksel bir eklentisiydi ve o günün şartlarına göre oldukça ilkel sayılırdı.) Nezaket Windows' u kritik bir şekilde eleştirmemize engel oluyor, ama Windows' un bir insanın işletim sisteminden beklediği her şeyi, grafik kullanıcı arayüzü ve çoğu cihazı destekleme özelliği gibi, ortalama olarak karşıladığını söyleyebiliriz.

Apple’ın Macintosh' u 1984' te piyasaya sürüldü ve o zamandan beri Mac OS olarak adlandırılan bir işletim sistemini kullanıyor. Yıllardan beri, Apple tabanda birçok değişiklik yaptı (günümüzün Mac' leri teknik olarak Windows bilgisayarlar ile aynı) ve bu değişikliklerin bazısı oldukça radikal oldu. 9 numaralı sürümüne kadar (9 da dahil olmak üzere ) Mac OS zayıf bir yapıydı, örneğin aynı anda çalışan birden fazla programa yetersiz derecede destek olabiliyordu. Günümüzdeki Mac OS X deki X bir Romen “10” rakamı (X karakteri değil) ve bu sistem temelde BSD Unix özelliklerini taşıyor.

Şubat 2012' den beri, Macintosh işletim sisteminin resmi ismi “Mac OS X” den “OS X” e değişti. Eğer “Mac OS” un gitmesine izin verirsek ne demek istediğimiz daha iyi anlaşılabilir.

Windows ve OS X arasındaki büyük fark, OS X' in Apple bilgisayarları ile özel olarak satılması ve “normal” bilgisayarlarda çalışmıyor olması. Bu durum Apple' a çok kararlı bir sistem oluşturmayı sağlıyor. Bunun dışında Windows neredeyse bütün bilgisayarlarda çalışıyor ve hiç görülmeyecek şekillerde birleşebilecek çok geniş bir donanıma destek sunuyor. Bu yüzden Windows kullanıcıları bazen çözülmesi zor ya da bazen çözülmesi imkansız uyumsuzluk sorunları ile karşılaşabilirler.  Buna rağmen Windows tabanlı bilgisayarlarda çok geniş bir donanım seçme imkanı bulunur ve böylece fiyatlar daha uygun seviyeye iner.

Windows ve OS X' in benzer noktaları ikisininde patentli bir yazılım olması: Kullanıcılar Microsoft ya da Apple önlerine neyi koyarsa onu kabul etmek zorundalar ve değil sistemi değiştirmek sistemin kendisini inceleyemezler bile. Bir güncelleme sistemine bağlıdırlar ve eğer üretici bir şeyi siler ya da başka bir şey ile değiştirirse buna uyum sağlamak zorundadırlar.

Burada bir fark bulunur: Apple genel olarak bir donanım üreticisidir ve OS X' i sadece Mac bilgisayarları alanlara sağlar. (bu yüzden OS X Mac-olmayan bilgisayarlar için değildir.) Diğer elde Microsoft bilgisayarları inşa etmez, sadece bilgisayarlarda çalışacak Windows gibi yazılımları satarak para kazanır. Bununla birlikte, Linux gibi bir işletim sistemi Apple' dan çok Microsoft' a bir tehdit oluşturur çünkü Apple alan insanların çoğu Apple bilgisayarı (tüm paketi) istedikleri için alırlar sadece OS X ile ilgilendikleri için değil. Bilgisayar dünyası, tablet ve diğer yeni moda Windows çalıştırmayan bilgisayarların istilası altına girdi ve bu Microsoft' u büyük bir baskı altına sokuyor. Apple bu durumdan Mac' ler yerine iPhone ve iPad'leri satarak kolayca kurtulabilir ancak Windows olmadan Microsoft iflasın eşiğine gelebilir.
\footnote{Aslında gerçek savaş alanı Windows değil Office'dir – çoğu insan Windows' u hayranlıklarından kullanmaz, eğlence için kullanılabilen (ucuz) bir işletim sistemi olduğu için kullanır ve bu bilgisayarlar Office'i çalıştırır-- fakat aynısı Apple ve Google'ı yer değiştirirsek de olur. Bir gerçek olarak, Office ve Windows Microsoft'un para kazanabildiği yegane ürünlerdir; bunun dışındaki Microsoft'un yaptığı her şey (muhtemelen Xbox oyun konsolu hariç) bir kayıptır.}
\end{subsection}
\begin{subsection}{Linux}

Linux ilk başta Linus Torvalds tarafından bir merak duygusu ile başladı fakat sonra kendi hayatını kurmaya devam etti. Bu zamanlarda yüzlerce geliştirici (sadece öğrenci ve bu işle hobi için uğraşan insanlar değil ayrıca IBM, Red Hat ve Oracle gibi firmalardaki profesyoneller) tarafından geliştirildi.

Linux'ta 1970'lerde AT\&T nin Bell Laboratuvarında geliştirilen ve küçük (küçük tanımı için yukarıyı inceleyin) bilgisayarlar için tasarlanan bir işletim sistemi olan Unix' ten esinlenilmişti. Unix kısa sürede araştırma ve teknoloji alanında tercih edilen bir sistem olmuştu. Büyük bir bölümde Linux Unix' in tasarımını ve temel fikirlerini kullanmaktadır ve Unix yazılımını Linux üzerinde çalıştırmak oldukça kolaydır fakat Linux'un kendisi Unix kodlarını içermez ve bağımsız bir projedir. Windows ve OS X'in tersine Linux ekonomik olarak ayrı bir şirket tarafından desteklenmiyor. Linux özgürce alınabilir ve oyunu kuralına göre oynayan herkes tarafından kullanılabilir. (bir sonraki bölümde belirtildiği gibi) Bütün bunlarla birlikte Linux sadece kişisel bilgisayarlar üzerinde değil, telefonlardan (en popüler akıllı telefon işletim sistemi, Android, bir Linux türevidir) en büyük anaçatı bilgisayarlara (dünyanın en hızlı 10 bilgisayarının hepsi Linux üzerinde çalışır) kadar çoğu sistemde çalışır ve bu Linux' u modern bilgisayar tarihinde en esnek işletim sistemi yapar.

Linux tek başına bir işletim sistemi çekirdeğidir ,uygulamaların ve özelliklerin kaynak kullanımını ayarlayan bir programdır. Uygulamaları olmadan gelen bir işletim sistemi çok kullanışlı olmadığından, insanlar genelde bir Linux dağıtımını yükler. Dağıtım, kararlı bir Linux' a sahip belirli uygulama, özellik, belgeleme ve diğer kullanışlı özellikleri barındıran bir pakettir. Buradaki güzel olan şey, Linux' un kendisi gibi, çoğu Linux dağıtımı özgürce kullanılabilir bir durumdadır ve ücretsiz ya da çok düşük bir fiyatla kullanılabilirdir. Bu şekilde binlerce dolarlık Windows ve OS X lisansı almadan ve lisans kısıtlamalarına maruz kalmadan bütün bilgisayarlarınıza, Ayşe teyzenizin ve arkadaşlarınız Zeynep ve Emre' nin bilgisayarlarına rahatça Linux dağıtımlarından istediğinizi kurabilirsiniz.

Linux ve Linux dağıtımları ile ilgili daha fazla bilgiyi ikinci bölümde bulabilirsiniz.
\end{subsection}
\begin{subsection}{Farklılıklar ve Benzerlikler}

Aslında bu üç büyük işletim sistemi sadece kullanıcıya sundukları şeyler konusunda sadece detaylarda farklılık gösterirler. Bu üç sistemde kullanıcıya herkesin kullanabileceği şekilde dosyaları “sürükle ve bırak” şeklinde özelliklere sahip olan bir grafiksel kullanıcı arayüzü sunar. Çoğu popüler uygulamalar bu üç sistem içinde mevcuttur, bu yüzden zamanınızın çoğunu internet tarayıcısı, ofis uygulamaları ya da bir e-mail programında harcadığınız için aslında hangisini kullandığınızın bir önemi kalmaz. Bu bir avantajdır çünkü bu bir sistemden diğerine geçişi mümkün kular.

Grafiksel arayüz dışında, üç sistemde metin şeklinde girilen komutların sistem tarafından çalıştırılabileceği bir çeşit “komut satırı” sunar. Windows ve OS X' de bu özellik genelde sistem yöneticileri için kullanılabilir durumdadır, normal bir kullanıcının bunu kullanmaması gerektiği yönünde bir düşünce bulunur. Linux' ta komut satırı daha az saklanmış vaziyettedir, bu Unix'in bilimsel/teknolojik felsefesi nedeniyle olabilir. Aslında bir gerçek olarak, çoğu görev Linux' un(ve bir açıdan da OS X' in) sağladığı güçlü araçlarla birlikte komut satırından daha etkili bir şekilde çalıştırılabilir. Yeni bir Linux kullanıcısı olarak sizde komut satırını açıp, bunun güçlü ve zayıf yönlerini öğrenmelisiniz, grafiksel arayüzün güçlü ve zayıf yönlerini öğrenmeniz gerektiği gibi. İkisinin karışımı size oldukça büyük bir çok yönlülük kazandıracaktır.
\paragraph{Alıştırmalar}{
\begin{itemize}
 \item Eğer Windows ve OS X gibi patentli bir işletim sistemi ile deneyiminiz olduysa: En sık hangi uygulamaları kullanıyorsunuz? Bunlardan hangileri özgür yazılım?
\end{itemize}}
\end{subsection}
\end{section}

\begin{section}{Özet}

Bugünün kişisel bilgisayarları, ister Linux tabanlı olsun ister Windows ya da OS X donanımları, temel konseptleri ve kullanım amaçları düşünüldüğünde farklılıktan çok benzerliklere sahipler. Hiç şüphe duymadan günlük işlerinizi yapmak için bu üç sistemden birini kullanabilirsiniz, hiçbiri açıkça ve itiraz edilemez bir şekilde "en iyi" değil. Bununla birlikte, bu kitap daha çok Linux hakkında, kalan sayfalar boyunca sistemin kullanımı ile ilgili olabilecek en fazla bilgiyi sağlamaya çalışacağız, sistemin gücüyle ilgili olan kısımları ortaya koyacağız ve zayıf olan yönlerinden bahsedeceğiz. Şimdilik Linux diğer iki sisteme karşı ciddi bir alternatif ve diğerlerinden değişik açılardan -özellikle bazı yönlerden- daha üstün. Sizin Linux ile ilgilendiğinizi görmekten mutluyuz ve umarız Linux'u öğrenirken, pratik yaparken ve kullanırken eğlenirsiniz. Eğer LPI'nin Temel Seviye Linux sertifikası ile ilgileniyorsanız sınavda başarılar dileriz!
\paragraph{Özet}{
\begin{itemize}
\item Bilgisayarlar verileri otomatik olarak çalıştıran, koşullu çalışmaya ve döngülere izin veren, değiştirilebilir programları barındıran cihazlardır.
\item Bilgisayarın en önemli parçaları işlemci, bellek, ekran kartı, anakart, sabit disk ve buna benzer parçalardır. 
\item Bilgisayarda bulunan yazılımlar firmware, işletim sistemi ve kullanıcı seviyesi programlar olarak üç gruba ayrılabilir. 
\item En popüler işletim sistemi Microsoft Windows' tur. Apple bilgisayarları OS X olarak adlandırılan başka bir işletim sistemi kullanır.
\item Linux bilgisayarlar için alternatif bir işletim sistemidir ve tek bir şirket tarafından değil çok sayıda gönüllü tarafından geliştirilir. 
\item Linux dağıtımları Linux işletim sistemi çekirdeğini uygulamalar ve belgelendirmeler ile kullanılabilir bir sisteme dönüştürür
\end{itemize}}
\end{section}

