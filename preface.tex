\chapter*{Önsöz}

Linux Essentials is a new certification by the Linux Professional Institute (LPI) which is aimed especially at schools and universities in order to introduce children and young adults to Linux. The Linux Essentials certificate is slated to define the basic knowledge necessary to use a Linux computer productively, and through a cor- responding education programme aid young people and adults new to the open source community in understanding Linux and open-source software in the con- text of the ITC industry. See appendix C for more informaton about the Linux Essentials certificate.

With this training manual, Linup Front GmbH introduces the first comprehen- sive documentation for Linux Essentials exam preparation. The manual presents the requisite knowledge extensively and with many practical examples and thus provides candidates, but also Linux newcomers in general, with a solid founda- tion for using and understanding the free Linux operating system as well as for attaining in-depth knowledge about running and administering Linux. In addi- tion to a detailed introduction to the background of Linux and free/open-source software, we explain the most important Linux concepts and tools such as the shell, how to handle files and scripts, and the file system structure. Insights into system administration, user and permission management and Linux as a network- ing client round off the presentation.

Based on the content of this training manual, Linux Essentials alumni are well- prepared to pursue further certifications including the LPI’s LPIC programme as well as vendor-specific certificates like those from Red Hat or Novell/SUSE.

The training manual is particularly suitable for a Linux Essentials preparation class at general-education or vocational schools, academies, or universities, but by virtue of its detailed approach and numerous exercises with sample solutions can also be used for self-study.

This courseware package is designed to support the training course as effi- ciently as possible, by presenting the material in a dense, extensive format for reading along, revision or preparation. The material is divided in self-contained chapters detailing a part of the curriculum; a chapter’s goals and prerequisites are summarized clearly at its beginning, while at the end there is a summary and (where appropriate) pointers to additional literature or web pages with further information.

Additional material or background information is marked by the “light- bulb” icon at the beginning of a paragraph. Occasionally these paragraphs make use of concepts that are really explained only later in the courseware, in order to establish a broader context of the material just introduced; these “lightbulb” paragraphs may be fully understandable only when the course- ware package is perused for a second time after the actual course.

Paragraphs with the “caution sign” direct your attention to possible prob- lems or issues requiring particular care. Watch out for the dangerous bends!

Most chapters also contain exercises, which are marked with a “pencil” icon at the beginning of each paragraph. The exercises are numbered, and sample solutions for the most important ones are given at the end of the courseware package. Each exercise features a level of difficulty in brackets. Exercises marked with an exclamation point (“\!”) are especially recommended.

Excerpts from configuration files, command examples and examples of computer output appear in typewriter type. In multilinedialogsbetweentheuserand the computer, user input is given in bold typewriter type in order to avoid misun- derstandings. The “<<<<” symbol appears where part of a command’s output had to be omitted. Occasionally, additional line breaks had to be added to make things fit; these appear as “<<”. When command syntax is discussed, words enclosed in angle brack- ets (“<Word>”) denote “variables” that can assume different values; material in brackets (“[-f <file>]”) is optional. Alternatives are separated using a vertical bar (“-a|-b”).

Important concepts are emphasized using “marginal notes” so they can be eas- ily located; definitions of important terms appear in bold type in the text as well as in the margin.

References to the literature and to interesting web pages appear as “[GPL91]” in the text and are cross-referenced in detail at the end of each chapter.

We endeavour to provide courseware that is as up-to-date, complete and error- free as possible. In spite of this, problems or inaccuracies may creep in. If you notice something that you think could be improved, please do let us know, e.g., by sending e-mail to
\begin{verbatim}
courseware@linupfront.de
\end{verbatim}
(For simplicity, please quote the title of the courseware package, the revision ID on the back of the title page and the page number(s) in question.) We also welcome contact by telephone, telefax or “snail mail”. Thank you very much!