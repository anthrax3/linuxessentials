\chapter{Kullanıcı Yönetimi}
\paragraph{Amaçlar}{
\begin{itemize}
 \item 
 \end{itemize}}
\paragraph{Önceden Bilinmesi Gerekenler}
\begin{itemize}
 \item 
 \end{itemize}

\begin{section}{Temeller}

\begin{subsection}{Neden Kullanıcılar?}

\end{subsection}
\begin{subsection}{Kullanıcı ve Gruplar}

\end{subsection}
\begin{subsection}{İnsan ve Pseudo-kullanıcılar}

\end{subsection}

\end{section}



\begin{section}{Kullanıcı ve Grup Bilgisi}

\begin{subsection}{/etc/passwd dosyası}

\end{subsection}
\begin{subsection}{/etc/shadow dosyası}

\end{subsection}
\begin{subsection}{/etc/group dosyası}

\end{subsection}
\begin{subsection}{/etc/gshadow dosyası}

\end{subsection}

\end{section}


\begin{section}{Kullanıcı Hesaplarını ve Grup Bilgisini Yönetmek}
\begin{subsection}{Kullanıcı Hesabını Oluşturmak}

\end{subsection}
\begin{subsection}{passwd Komutu}

\end{subsection}
\begin{subsection}{Kullanıcı Hesabını Silmek}

\end{subsection}
\begin{subsection}{Kullanıcı Hesabını ve Grup Bilgisini Değiştirmek}

\end{subsection}
\begin{subsection}{Kullanıcı Bilgisini Doğrudan Değiştirmek --- vipw}

\end{subsection}
\begin{subsection}{Grup Oluşturmak, Değiştirmek ve Silmek}

\end{subsection}

\end{section}